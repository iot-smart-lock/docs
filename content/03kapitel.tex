%!TEX root = ../dokumentation.tex

\chapter{Ausblick und Fazit}
Final lassen sich noch Überlegungen anstellen, wie das entwickelte Produkt verbessert werden könnte und welche Entwicklungsschritte notwendig wären, um eine Produktion im großem Umfang zu beginnen. 

Ein relevantes und bisher kaum betrachtetes Thema ist die IT-Sicherheit. Im umgesetzten Rahmen wurden sowohl bei der Bluetooth-Verbindung als auch bei der Datenübertragung über \ac{LoRa} keine Verschlüsselungen betrachtet. Um das Trial der IT-Sicherheit von Vertraulichkeit, Integrität und Verfügbarkeit zu erfüllen, müsste das Thema Verschlüsselung näher betrachtet und umgesetzt werden.

Auch Updates wurden bisher ungeachtet gelassen. Sogenannte \ac{OTA}-Updates sind mittlerweile Standard und erfordern erhöhte Datenübertragung für einen kurzen Zeitrahmen. Um dies zu ermöglichen, müsste sowohl das Übertragungsprotokoll, als auch die Verfügbarkeit während den Updates betrachtet werden.

Für die reale Umsetzung und Produktion wäre selbstverständlich auch die Ingenieursleistung der Hardware notwendig. Während im \ac{IoT}-Labor lediglich ein \ac{PIO}-Pin zur Signalanzeige verwendet wurde, müsste für die Produktion der Magnetauslöser des elektrischen Schlosses entwickelt werden und der Chip samt Sensoren in ein Vorhängeschloss eingebaut werden. Erforderliche Schritte für eine Produktion wären zudem Themen wie Pricing, legislative Bestimmungen, Verantwortlichkeiten, Support und weitere administrative Eigenschaften.

Abschließend lässt sich dennoch sagen, dass das \ac{IoT}-Labor der Studenten ein Erfolg war. Die Studenten beschäftigten sich mit der ESP32-Chip-Programmierung, der Netzwerkeinrichtung, der Node-Red-Konfiguration und der Applikationsentwicklung, um den entstandenen Prototypen eines Smart-Lock zu entwickeln.