%!TEX root = ../dokumentation.tex

\chapter{Einleitung / Business Case}
Im Rahmen des IoT-Labs widmeten sich die Studenten einem ganz bestimmten dieser Probleme: Vorhängeschlösser. Vorhängeschlösser kommen an vielen Orten zum Einsatz: zum Gartenhäuschen Verriegeln, in der Umkleide des Fitnessstudios, zum Fahrrad Abschließen, zum Garagentor Verriegeln und vielen anderen Einsatzgebieten. Diese Vorhängeschlösser bringen dabei allerdings ein ganz spezifisches Problem mit sich. Vorhängeschlösser werden mit einem Schlüssel ge- und entsperrt und haben damit ein hohes Verlustpotential. Die Schlüssel sind meist sehr klein und werden in einigen Fällen auch nicht häufig benötigt (Gartenhäuschen nur im Sommer). Eine weitere Unannehmlichkeit ist beim Anwendungsfall des Fitnessstudios zu finden. Die Kabine wird meist lediglich mit einer Flasche für die Versorgung, einem Handtuch für die Übungsausführung und dem Handy zur musikalischen Begleitung oder anderem Entertainment während dem Training verlassen. Jetzt ist da aber noch der Spindschlüssel. Der passt irgendwie nicht ins Muster. In der Tasche, wenn die Sportbekleidung eine solche hat, ist der Schlüssel sehr unangenehm. Den Schlüssel in der Handyhülle zu verstauen ist meist nicht möglich, dafür ist der Schlüssel schlicht zu dick. Und da sonst keine Gegenstände mitgeführt werden, bietet sich keine vernünftige Verstauungsmöglichkeit für den Schlüssel an! 
\\ 
Hier kommt die IoT-Lösung Smart-Lock ins Spiel. Smart-Lock soll ein Vorhängeschloss sein, welches zuverlässig über eine Handy-App ver- und entriegelt werden kann. Dies behebt die Probleme des Schlüsselverlusts und der \glqq Kruschtelei\grqq{} mit den kleinen Schlüsseln.
\\ 
Wie die IoT-Lösung des intelligenten Schlosses umgesetzt wurde, was es für Funktionalitäten hat und wo eventuelle Hindernisse liegen, soll im Folgenden aufgeführt werden.