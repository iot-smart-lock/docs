%!TEX root = ../../dokumentation.tex

\section{Network-Layer} \label{network}
Das Network-Layer beschreibt die Protokolle und Netzwerk-Server, die die \ac{IoT}-Lösung zum Austausch der Daten verwendetet. Dabei wird in der vorgestellten Lösung auf vier verschiedene übertragungstechnologien gesetzt.

Für die Übertragung der \ac{GPS}-Daten wird \textbf{\ac{LoRaWAN}} verwendet und sendet die Daten an ein Gateway. Das Gateway ist Teil des \ac{TTN}s, ein zusammenschluss aus vielen öffentlichen \ac{LoRaWAN}-Gateways. Im \ac{TTN} werden die Daten über einen Parser in ein \ac{JSON}-Format gebracht und können über herkömmliche Netzwerkprotokolle, wie \ac{HTTP}-WebSockets oder \ac{MQTT} an einen Server übertragen werden.

Die vorgestellte Smart Lock Lösung verwendet für die weitere Übertragung an das Service-Layer \textbf{\ac{MQTT}}. Das hat den Vorteil, dass sich Nachrichten sobald sie Empfangen werden, an den Server übertragen lassen und nicht über ein Request-Response verfahren abgefragt werden müssen. 

Zusätzlich zu \ac{LoRaWAN} und \ac{MQTT} wird in der Lösung \textbf{\ac{HTTP}} für die Kommunikation von App und Server verwendet. Hierfür sendet die Benutzer-App eine \ac{HTTP}-Request an den Node-Red Server, der die Anfrage verarbeitet und die Device-Daten an die App zurücksendet. 

Zuletzt wird \textbf{\ac{BLE}} für das Öffnen und Schließen des Schlosses und somit für die direkte Kommunikation von App zu Device verwendet. \ac{BLE} hat zwei entscheidende Vorteile: geringer Energieverbrauch und geringe Reichweite. Ersteres ist notwendig, damit das Schloss möglichst lange Akkulaufzeiten hat und zweiteres damit das Schloss nur aus unmittelbarer Nähe geöffnet oder geschlossen werden kann.

Zusammengefasst, deckt das Network-Layer die Kommunikation von allen verwendeten Hardware- und Software-Modulen ab und verwendet hierfür \ac{LoRaWAN} über das \ac{TTN} als Gateway zu \ac{MQTT}, sowie \ac{HTTP} und \ac{BLE} für die Kommunikation zur Benutzer-App.