%!TEX root = ../../dokumentation.tex

\section{Device-Layer}

Im Device-Layer finden sich alle Sensoren und Aktoren der \ac{IoT}-Lösung wieder. 
Zu den Sensoren gehört ein \ac{GPS}-Chip, der \ac{GPS}-Daten empfängt, um den Standort des Smart-Locks festzustellen. 

\ac{GPS}-Daten können in verschiedenen Darstellungsoptionen empfangen werden, darunter \emph{GPGGA} und \emph{GPRMC}. Da sich die empfangen Daten im Aufbau 
je nach Protokoll unterscheiden, muss eine Vorverarbeitung der Daten stattfinden, um genaue Positionsdaten zu erhalten.
Hierfür werden die \ac{GPS}-Daten, die in Form eines \emph{Strings} empfangen werden, anhand der im Datenstring enthaltenen Kommatas getrennt.
Um nun für Karten verwendbare Positionsdaten zu erhalten, müssen die ausgelesenen Positionsdaten in Grad und Minuten konvertiert werden. Nachdem diese 
Daten vorverarbeitet wurden sind diese Sendebereit.

Eine \ac{LED}, die den Zustand des Smart-Locks, also ob geschlossen oder offen, darstellt, gehört zur Gruppe der Aktoren. Diese ist mit Hilfe eines \ac{PIO}-Pin direkt an den Mikrokontroller angeschlossen.

Des Weiteren befinden sich zwei Netzwerkschnittstellen in Form zwei gesonderter Chips im Device-Layer der \ac{IoT}-Lösung. Ein \ac{LoRa}-Chip dient dem Senden und dem Empfangen von 
\ac{LoRa}-Nachrichten mit Hilfe von \ac{LoRaWAN} an das \ac{TTN}. Ein \ac{BLE}-Chip ermöglicht eine Verbindung mit einem Mobilgerät und dient dem Empfangen von Befehlen, 
die den Zustand des Smart-Locks ändern können. Diese Netzwerkprotokolle werden in Kapitel \ref{network} genauer beschrieben.

Die Stromversorgung ist durch einen eingebauten Akku auf der Platine des Mikrokontrollers sichergestellt.
