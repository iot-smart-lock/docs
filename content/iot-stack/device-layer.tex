%!TEX root = ../../dokumentation.tex

\section{Device-Layer}

Im Device-Layer finden sich alle Sensoren und Aktoren der \ac{IoT}-Lösung wieder. 
Zu den Sensoren gehört ein \ac{GPS}-Chip, der \ac{GPS}-Daten empfängt, um den Standort des Smart-Locks festzustellen. Eine \ac{LED}, die den Zustand des Smart-Locks, also ob geschlossen 
oder offen, darstellt, gehört zur Gruppe der Aktoren.

Des Weiteren befinden sich zwei Netzwerkschnittstellen in Form zwei gesonderter Chips im Device-Layer der \ac{IoT}-Lösung. Ein \ac{LoRa}-Chip dient dem Senden und dem Empfangen von 
\ac{LoRa}-Nachrichten mit Hilfe von \ac{LoRaWAN} an das \ac{TTN}. Ein \ac{BLE}-Chip ermöglicht eine Verbindung mit einem Mobilgerät und dient dem Empfangen von Befehlen, 
die den Zustand des Smart-Locks ändern können. Diese Netzwerkprotokolle werden in Kapitel \ref{network} genauer beschrieben.