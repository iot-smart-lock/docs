%		\ac{Abk.}   --> fügt die Abkürzung ein, beim ersten Aufruf wird zusätzlich automatisch die ausgeschriebene Version davor eingefügt bzw. in einer Fußnote (hierfür muss in header.tex \usepackage[printonlyused,footnote]{acronym} stehen) dargestellt
%		\acf{Abk.}   --> fügt die Abkürzung UND die Erklärung ein
%		\acl{Abk.}   --> fügt nur die Erklärung ein
%		\acp{Abk.}  --> gibt Plural aus (angefügtes 's'); das zusätzliche 'p' funktioniert auch bei obigen Befehlen
%		\acs{Abk.}   -->  fügt die Abkürzung ein
%	siehe auch: http://golatex.de/wiki/%5Cacronym
%!TEX root = ../dokumentation.tex
%nur verwendete Akronyme werden letztlich im Abkürzungsverzeichnis des Dokuments angezeigt
%Verwendung: 
\acro{BLE}{Bluetooth Low Energy}
\acro{GPS}{Global Positioning System}
\acro{HTTP}{Hypertext Transfer Protokol}
\acro{IoT}{Internet of Things} % Beispielabkürzung
\acro{JSON}{JavaScript Objective Notation}
\acro{LED}{Light Emitting Diode}
\acro{LoRa}{Long Range (Low Power)}
\acro{LoRaWAN}{Long Range Wide Area Network} 
\acro{MQTT}{Message Queuing Telemetry Transport}
\acro{OTA}{Over-the-Air}
\acro{PIO}{Programmed Input/Output}
\acro{TTN}{The Things Network}